
\documentclass{dima}
          
\begin{document}

% ****************** TITLE ****************************************

\title{Data Analytics using {\ttlit KNIME} open source tool}

% ****************** AUTHORS **************************************

\numberofauthors{3}

\author{
% 1st. author
\alignauthor{Dörheit, Eric\\
       \affaddr{TU Berlin}\\
       \affaddr{Berlin, Germany}\\
       \email{eric.doerheit@campus.tu-berlin.de}}
% 2nd. author
\alignauthor{Bode, Olga\\
       \affaddr{TU Berlin}\\
       \affaddr{Berlin, Germany}\\
       \email{olga.bode@mailbox.tu-berlin.de}}
% 3rd. author
\alignauthor{Poljak, Dorothea\\
       \affaddr{TU Berlin}\\
       \affaddr{Berlin, Germany}\\
       \email{@mailbox.tu-berlin.de}}
}

\maketitle

% ****************** TEXT **************************************

\begin{abstract}
We present our results of the evaluation of the open source tool \textit{KNIME} which is used for data analytics and data mining. We choose anomaly detection as the subject to evaluate \textit{KNIME} as many methods of data analytics such as clustering, classification, time series analysis and statistical techniques are applicable to anomaly detection \cite{Chandola:2009:ADS:1541880.1541882}. As a data set for the analysis we use the data provided for the \textit{DEBS Grand Challenge 2012} \cite{Jerzak:2012:DGC:2335484.2335536}.
\end{abstract}

\section{Introduction}
For the evaluation of \textit{KNIME} we first investigated the tool by following several white-papers provided by \textit{KNIME} \footnote{Explain contents of white-papers}. We start with describing the functionalities and the usage of \textit{KNIME}. Then we explain the data used for the evaluation. Subsequently we provide an overview on anomaly detection based on \cite{Chandola:2009:ADS:1541880.1541882}. In Section~\ref{sec:ADwK} we evaluate \textit{KNIME} through applying anomaly detection on the data set described in Section~\ref{sec:D2012GC}.

\subsection{The Open Source Tool KNIME}

\subsection{DEBS 2012 Grand Challenge}
\label{sec:D2012GC}

\subsection{Anomaly Detection}
\subsubsection{Based on Classification}
\subsubsection{Based on Nearest Neighbor}
\subsubsection{Based on Classification}
\subsubsection{Statistical Anomaly Detection Techniques}

\section{Anomaly Detection with KNIME}
\label{sec:ADwK}

\subsection{Clustering}
\subsection{Classification}

\section{Evaluation}

\section{Conclusions}

\bibliographystyle{abbrv}
\bibliography{sigproc}

\end{document}
